\documentclass[11pt,a4paper,svgnames]{article}
\usepackage{babel}
\usepackage{fontspec}
\usepackage{luacode}
\usepackage{fancyhdr}
\usepackage{relsize}
%\usepackage[useregional]{datetime2}
%see https://tex.stackexchange.com/a/317859/42406
%\usepackage[utf8]{luainputenc}
\usepackage[T1]{fontenc}
%\usepackage{xcolor}
\usepackage{hyperref}
%\usepackage{alltt}
%\usepackage{wasysym}
\usepackage{xcolor}
\usepackage{times}
%\usepackage[square]{natbib}
\usepackage{textcomp}
\usepackage{setspace}
\usepackage[backend=biber,bibencoding=utf8]{biblatex}
\addbibresource{../refpersys-bib.bib}
\usepackage[a4paper, margin=4cm]{geometry}
\setstretch{1.2}
\date{october 2019}
\newcommand*{\ac}[1]{\mbox{#1}}
\tolerance=600

\begin{luacode*}
  local gitpip=io.popen("git log --no-color --format=oneline -1 --abbrev=16 --abbrev-commit -q | cut -d' ' -f1")
  gitid=gitpip:read()
  gitpip:close()
  docdate = os.date('%Y-%b-%d %H:%M')
\end{luacode*}
\newcommand{\mygitid}{\luadirect{tex.print(gitid)}}
\newcommand{\mydate}{\luadirect{tex.print(docdate)}}
\newcommand{\RefPerSys}{{\textit{\textsc{RefPerSys}}}}

 % see https://tex.stackexchange.com/a/51349/42406
\hypersetup{
  colorlinks   = true, %Colours links instead of ugly boxes
  urlcolor     = NavyBlue, %Colour for external hyperlinks
  linkcolor    = DarkGreen, %Colour of internal links
  citecolor   = DarkMagenta, %Colour of citations
  frenchlinks = true,
}

\pagestyle{fancy}
\fancyhf{}
\rhead{\textit{\textsc{RefPerSys}} high-level goals and design ideas}
\fancyfoot[L]{{\raisebox{0.0cm}[1pt][1pt]{\color{LightSlateGrey}{{\relsize{+1}{DRAFT}}}~\relsize{-1.5}{\texttt{\mygitid}}}}}
\fancyfoot[R]{{Page \thepage}}
\begin{document}
\title {\textit{\textsc{RefPerSys}} high-level goals and design
  ideas\thanks{git id \texttt{\mygitid}, generated on \textit{\mydate}}}
\author {Basile
  \textsc{Starynkevitch}\thanks{See \href{http://starynkevitch.net/Basile/}{\texttt{starynkevitch.net/Basile/}}
    and contact
    \href{mailto:basile@starynkevitch.net}{\texttt{basile@starynkevitch.net}},
    92340 Bourg La Reine {\relsize{-1}(near Paris)}, France.}}

\begin{titlepage}
  \thispagestyle{empty}
  \maketitle

  \bigskip


  \tableofcontents
  
\end{titlepage}



\section{design goals}

The \RefPerSys\footnote{For a \textbf{Ref}lexive \textbf{Per}sistent
\textbf{Sys}tem}~ system share several -but not all- goals and design
ideas (but no code) with
\href{http://github.com/bstarynk/bismon}{\texttt{bismon}}
\cite{Starynkevitch:2019:bismon-draft} but of course \emph{not}
\texttt{bismon}'s application to
\href{https://en.wikipedia.org/wiki/Static_program_analysis}{static
  source code analysis}. \textbf{\RefPerSys~ is a long term\footnote{I
  don't expect any significant research results before $\approx
  2025$.}  risky
  \href{https://en.wikipedia.org/wiki/Research}{research} project and
  a \href{https://www.gnu.org/philosophy/free-sw.en.html}{free
    software} licensed under
  \href{https://www.gnu.org/licenses/gpl-3.0.html}{GPLv3+}, and
  targetted \emph{only} for \textsc{Linux x86-64} computers.}. A Linux
system\footnote{My own \texttt{ours.starynkevitch.net} computer,
running \textit{Debian/Unstable}, has 64 Gibytes of RAM, 24 cores (AMD
2970WX) and terabytes of disk space, including a terabyte of SSD.}
with at least 16 Gibytes of RAM, 4 cores, and 120 Gibytes of disk is
required. The grand ambition of {\RefPerSys} is to become later an
infrastructure for some strong
\href{https://en.wikipedia.org/wiki/Artificial_general_intelligence}{AGI}
system à la \textsc{Caia} by Jacques Pitrat\footnote{Jacques Pitrat
has passed away on october 14\textsuperscript{th}, 2019.}
\cite{Pitrat:1996:FGCS, Pitrat:2009:AST, Pitrat:2009:ArtifBeings}, but
before even approaching that goal a big lot of work is required, and
{\RefPerSys} should be valuable by itself for other less ambitious and
more pragmatical purposes.

The development of {\RefPerSys} is (like the one of \texttt{bismon}) a
slow and gradual process. Features added to {\RefPerSys} in January
2020 are used to implement new features worked on a later {\RefPerSys}
in March 2020.

As every practical software, {\RefPerSys} targets some defined
machines: common Linux distribution running on some
computer\footnote{For several years, that computer is a desktop or
powerful laptop running some \textsc{Debian}. Later that could be some
``virtual machine'' e.g. some
\href{https://www.docker.com/}{\textsc{Docker}} container.}. So the
target machine of {\RefPerSys} is a quite complete and modern Linux
system, with many useful packages, and administred by some human
person\footnote{For obvious cybersecurity reasons, automatic
administration of that Linux distribution is out of scope.}. The
{\RefPerSys} is published in ``source'' form, as a set of
\href{http://git-scm.com/}{\texttt{git}} versionned\footnote{We
crucially depend upon \texttt{git} specifically
(e.g. \href{http://gitlab.org/}{\texttt{gitlab}}), and porting to some
other versionning system would be a difficult task.} textual files
(e.g. C++ files -most and more and more\footnote{Of course, in a
\href{https://en.wikipedia.org/wiki/Chicken_or_the_egg}{chicken and
  egg} fashion, the initial version of {\RefPerSys} has to contain
mostly hand-written files!} of them being generated- or shell files or
data files). Some of these files are generated, and the bootstrapping
goal is to have \emph{every} \texttt{git}-registered textual file been
generated by {\RefPerSys}, with a
\href{https://en.wikipedia.org/wiki/Bootstrapping\_(compilers)}{\textbf{bootstrap}ed}
approach similar to those of
\href{https://en.wikipedia.org/wiki/Self-hosting_(compilers)}{self-hosting
  compilers}.
\medskip


\clearpage


\printbibliography

\end{document}

%%%%%%%%%%%%%%%%%%%%%%%%%%%%%%%%%%%%%%%%%%%%%%%%%%%%%%%%%%%%%%%%
%% Local Variables: ;;
%% compile-command: "./build.sh" ;;
%% End: ;;
%%%%%%%%%%%%%%%%%%%%%%%%%%%%%%%%%%%%%%%%%%%%%%%%%%%%%%%%%%%%%%%%
